Sobre los tipos de excepciones

Primera Excepcion Investigada

Para esta primera excepcion, lo que se realiza es que la excepcion de logica para poder lanzar una excepcion teniendo un mensaje personalizado que para el caso de que ambas variables de tipo booleanas sean verdaderas

Segunda Excepcion Investigada

Para la segunda excepcion, se tiene la excepcion de longitud para detectar el acceso de un fuera de rango sobre un elemento en la una palabra o cadenas de caracteres

Tercera Excepcion Investigada

Por ultimo, se usa el tipo de excepcion de dominio para poder detectar el error en el proceso matematico de realizar la division de un numero entero por cero.

Cabe destacar que se tiene el codigo \mbox{\hyperlink{exceptions_8cpp}{exceptions.\+cpp}} comentado como se solicita. Tambien se crea \mbox{\hyperlink{exceptions__main_8cpp}{exceptions\+\_\+main.\+cpp}} y \mbox{\hyperlink{exceptions_8hpp}{exceptions.\+hpp}} la cual representa la division de archivos solicitada de \mbox{\hyperlink{exceptions_8cpp}{exceptions.\+cpp}}

Ademas se agrega solo, el codigo \mbox{\hyperlink{tipos__excepciones_8cpp}{tipos\+\_\+excepciones.\+cpp}}, que solo presenta las excepciones investigadas

Instrucciones de Ejecucion por medio del Makefile, como son dos ejecuciones se dividen para mayor orden

Para la division de archivos de encabezado y fuente, donde solo se presenta el error de excepcion

make exceptions

Para ejecutar los tres tipos de excepciones investigadas make tipos 