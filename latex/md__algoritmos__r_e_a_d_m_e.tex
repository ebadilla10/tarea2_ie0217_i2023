Complejidad de los Algoritmos de Ordenamiento usando notacion Big O.


\begin{DoxyEnumerate}
\item Bubble\+Sort Para el algoritmo Bubble\+Sort se caracteriza por ser un algoritmo de ordenamiento sencillo y basico, donde se destaca por realizar la iteracion a traves de un arreglo varias veces, esto comparando los elementos adyacentes y realizando swaps hasta que el arreglo este ordenado. En relacion a la complejidad de tiempo de Bubble\+Sort es O(n$^\wedge$2), la razon es porque realiza dos bucles anidados de tamaño n, lo que resulta que n$\ast$n operaciones de comparacion y swap en el peor de los casos. Por lo tanto, se concluye que la complejidad de espacio de Bubble\+Sort es O(1), ya que solo utiliza variables de tamaño constante independientemente del tamaño del arreglo.
\item Selection\+Sort Este tipo de algoritmo de ordenamiento, funciona buscando el minimo elemento en un sub arreglo, y luego colocandolo en su posicion correcta. Se repite este proceso hasta que todo el arreglo este ordenado. El tipo de complejidad que presenta este tipo de arreglo es de O(n$^\wedge$2), similar al arreglo anterior y de igual forma la complejidad de espacio de Selection\+Sort es O(1).
\item Insertion Sort En este caso es un algorito de ordenamiento que tiene como funcionalidad de recorrer un arreglo, tomando cada elemento y colocandolo en su posicion correcta en un sub arreglo ordenado y contiene la misma complejidad cuando se utilizan la notacion Big O que los dos algoritmos anteriores.
\item Quick\+Sort Es tipo de ordenamiento avanzado, funcionada dividiendo el arreglo en dos sub arreglos, uno con elementos mas pequeños que un pivote y otro con elementos mayores que el pivote, y ordenando de forma recursiva cada sub arreglo. La complejidad de tiempo de Quick\+Sort es O(n$\ast$log(n)) en el mejor y promedio de los casos y O(n$^\wedge$2) para el peor de los casos. La complejidad de espacio de Quick\+Sort es O(log(n)) en el peor de los casos, producido por la recursividad del algoritmo.
\end{DoxyEnumerate}

Instruccion para ejecucion del programa por medio del Makefile

make algoritmo 