Sobre Includes\+:

Los includes que se presentan en el codigo, todos son necesarios. Se tienen los siguientes.

\#include iostream \+: Se utiliza para la entrada y salida estandar, si es necesario \#include vector \+: Para poder utilizar vectores, por lo tanto si es necesario \#include algorithm \+: Si es necesario porque se usa para la funcion for\+\_\+each, o elementos de secuencia \#include stdexcept \+: Brinda en cierta forma clases, para poder controlar excepciones estandar, como por ejemplo out\+\_\+fo\+\_\+range \#include functional \+: Se usa para definir y trabajar con objetos

Y con el codigo brindado es necesario por utilizar.

Nueva Instancia\+:

Con respecto a la nueva instancia, se realizo una investigacion y se agrega una nueva instancia donde utiliza el template dado, pero para otro tipo de dato, donde tiene el uso de una plantilla de clase stack que almacena elementos de tipo string.

Para alguna sentencia, declaracion o instruccion\+: Al final del codigo se agrega, se forma que se presenta comentado tal como se solicita en el enunciado, lo que realiza es que en el codigo se crea un objeto my\+Stack, cuando la pila esta vacia lo que hace es desencadenar la excepcion out\+\_\+of\+\_\+range entonces para manejar esta excepcion se usa una estructura try y catch para poder capturar la excepcion y luego imprimir el mensaje de error.

Instruccion para ejecucion del programa por medio del Makefile

make template 